\begin{abstract}
    The architecture of a deep neural network is defined explicitly in terms of the number of layers, the width of each layer and the general network topology. Existing optimisation frameworks neglect this information in favour of implicit architectural information (e.g.~second-order methods) or architecture-agnostic distance functions (e.g.~mirror descent). Meanwhile, the most popular optimiser in practice---Adam---is based on heuristics. This paper builds a new framework for deriving optimisation algorithms that explicitly leverage neural architecture. The theory extends mirror descent to non-convex composite objective functions: the idea is to transform a Bregman divergence to account for the non-linear structure of neural architecture. Working through the details for deep fully-connected networks yields \textit{automatic gradient descent}: a first-order optimiser without any hyperparameters. Automatic gradient descent trains both fully-connected and convolutional networks out-of-the-box and at ImageNet scale. A PyTorch implementation is available at \url{https://github.com/jxbz/agd} and also in \cref{app:pytorch}. Overall, the paper supplies a rigorous theoretical foundation for a next-generation of architecture-dependent optimisers that work automatically and without hyperparameters.
\end{abstract}